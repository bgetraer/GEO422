\documentclass[11pt]{article}

% packages
\usepackage{amsmath}
\usepackage{amsfonts}
\usepackage{amssymb}
\usepackage{wasysym}
\usepackage{graphicx}
\usepackage{pslatex}
\usepackage{lscape}
\usepackage{rotating}
\usepackage[T1]{fontenc}
\usepackage[latin1]{inputenc}
\usepackage{longtable}
\usepackage[font=scriptsize,labelfont=bf]{caption}
\setlength{\LTcapwidth}{5.5 in}
\usepackage{url}
\usepackage{lastpage}
\usepackage{hyperref}
\usepackage{gensymb}
\usepackage[absolute]{textpos}
\usepackage{color}
\usepackage[export]{adjustbox}

% frontmatter
\title{Exploring Spectral Analysis: PSET~5}
\author{
        Benjamin Getraer \\
        GEO~422:~Data, Models, \& Uncertainty in the Natural Sciences\\
}
\date{\today}

% It's nice to have a header
\usepackage{fancyhdr}
\pagestyle{fancy}
 \rhead {\textcolor{blue}{\href{mailto:bgetraer@princeton.edu}{Benjamin Getraer}}, \thepage ~of
     \pageref{LastPage}}
 \lhead{\footnotesize \textsc{GEO 422, Lab 05}}
 \cfoot{}
 \renewcommand{\headrulewidth}{0.4pt} 

% Now setup margins and linespacing
\textwidth = 6.75 in
\textheight = 9.0 in
\addtolength{\voffset}{-0.04in} 
\oddsidemargin = 0.0 in
\evensidemargin = 0.0 in
\topmargin =  0.0 in
\headheight = 0.1 in
\headwidth = 6.5 in
\headsep = 0.15 in
\parskip = 0.025 in
\parindent = 0.0 in
\linespread{1.25}

\begin{document}
\maketitle

\begin{abstract}
This is the paper's abstract \ldots
\end{abstract}

\section{Introduction}
Introduction

\paragraph{Outline}
The remainder of this article is organized as follows.
Section~\ref{sec:synthetic} introduces the implications of sampling and aliasing.
Our new and exciting results are described in Section~\ref{results}.
Finally, Section~\ref{conclusions} gives the conclusions.

\section{Sampling and Recovery of Synthetic Signals}\label{sec:synthetic}
Nyquist sampling follows the theorem that for a given signal $f(\rm{x})$ composed of frequencies $\underline{\omega}$, $f(\rm{x})$ can be recovered using a sampling rate of $\frac{1}{\Delta\rm{x}}\ge2\times \omega_{\text{max}}$.

\begin{figure}[h!]
\centering
\includegraphics[height=0.6\linewidth]{/Users/benjamingetraer/Documents/Fall2017/GEO422/Figures/PSET5/Nyquist.pdf}
\caption{Sampling (in red) of the signal $f(\rm{x})=\cos(2\pi\times\rm{x}) +\sin(2\pi\times2\rm{x}) +\sin(2\pi\times5.6\rm{x})$ (in blue) at various sampling rates, where ``Nyquist Sampling'' is defined as a sampling rate of $\frac{1}{\Delta\rm{x}}=2\times5.6$.}
\label{fig:nyquist}
\end{figure}

\section{Recovery of Unknown Signals}\label{sec:unknown}
The recovery of unknown signals from measurements can be approached in a similar fashion, with the sampling rate significantly affecting the frequencies detected in the signal. For instance, Figure~\ref{fig:yearly} shows a time-series measured annually, while Figure~\ref{fig:decadal} shows a time-series of the same phenomena measured every 10 years. The general process of analysis for each time-series went as follows: 

\begin{enumerate}
\item The data $\underline{\textbf{d}}$ were centered about their mean $\mu$, and normalized by their variance $\sigma^2$ such that the normalized data $\underline{\textbf{d}}^{\prime}=(\underline{\textbf{d}}-\mu)/ \sqrt{\sigma^2}$.
\item The power-spectral-density of the $\underline{\textbf{d}}^{\prime}$ signal frequencies $\underline{\omega}$ were estimated by Fourier transform.
\item Periodic structures in the signal were estimated by the element-wise inversion of period $\mathrm{L}={\omega}^{-1}$.
\end{enumerate}


\begin{figure}[h!]
\centering
\includegraphics[height=0.6\linewidth]{/Users/benjamingetraer/Documents/Fall2017/GEO422/Figures/PSET5/years.pdf}
\caption{Analysis of time-series given in \href{http://geoweb.princeton.edu/people/simons/YEARLY.PLT}{\texttt{YEARLY.PLT}}. On top are the data $\underline{\textbf{d}}$ with a normalized axis on the right showing the dimensions of $\underline{\textbf{d}}^{\prime}$ (see Section~\ref{sec:unknown}). On bottom left is the power-spectral-density (PSD) of the $\underline{\textbf{d}}^{\prime}$ signal frequencies estimated by \href{https://www.mathworks.com/help/signal/ref/periodogram.html}{\texttt{PERIODOGRAM}}, using the default box-car window. Inversion of the frequency shows periodic structure within the data around the 10 and 100 year time-scales (bottom right).}
\label{fig:yearly}
\end{figure}

\begin{figure}[h!]
\centering
\includegraphics[width=0.5\linewidth]{/Users/benjamingetraer/Documents/Fall2017/GEO422/Figures/PSET5/decadal.pdf}
\caption{Analysis of time-series given in \href{http://geoweb.princeton.edu/people/simons/DECADAL.PLT}{\texttt{DECADAL.PLT}}. On top are the data $\underline{\textbf{d}}$ with a normalized axis on the right showing the dimensions of $\underline{\textbf{d}}^{\prime}$ (see Section~\ref{sec:unknown}). On bottom left is the power-spectral-density (PSD) of the $\underline{\textbf{d}}^{\prime}$ signal frequencies estimated by \href{https://www.mathworks.com/help/signal/ref/periodogram.html}{\texttt{PERIODOGRAM}}, using the default box-car window. Inversion of the frequency shows periodic structure within the data around the 10 and 100 year time-scales (bottom right).}
\label{fig:decadal}
\end{figure}


\begin{figure}[h!]
\includegraphics[width=0.5\linewidth]{/Users/benjamingetraer/Documents/Fall2017/GEO422/Figures/PSET5/windowfunc.pdf}
\caption{Examples of various windowing functions, over the domain [$-1$,$1$], of length 1000.}
\label{fig:windows}
\end{figure}

\begin{figure}[h!]
\includegraphics[width=0.5\linewidth]{/Users/benjamingetraer/Documents/Fall2017/GEO422/Figures/PSET5/windows.pdf}
\caption{Demonstration of various windowing functions, using \href{https://www.mathworks.com/help/signal/ref/periodogram.html}{\texttt{PERIODOGRAM}} and the \href{http://geoweb.princeton.edu/people/simons/YEARLY.PLT}{\texttt{YEARLY.PLT}} dataset. Note that the shape of the estimated PSD is different for each window function, but that all show peaks at frequencies $\omega=0.1$, $0.09$, $0.0117$ corresponding to periods of $\mathrm{L}=10$, $11.1$, $85.5$ years.}
\label{fig:windows}
\end{figure}



\section{Results}\label{results}
In this section we describe the results.

\section{Conclusions}\label{conclusions}
We worked hard, and achieved very little.

\end{document}
