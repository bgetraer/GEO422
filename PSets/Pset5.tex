\documentclass[11pt]{article}

\usepackage{amsmath}
\usepackage{amsfonts}
\usepackage{amssymb}
\usepackage{wasysym}
\usepackage{graphicx}
\usepackage{pslatex}
\usepackage{lscape}
\usepackage{rotating}
\usepackage[T1]{fontenc}
\usepackage[latin1]{inputenc}
\usepackage{longtable}
\usepackage[font=scriptsize,labelfont=bf]{caption}
\setlength{\LTcapwidth}{5.5 in}
\usepackage{url}
\usepackage{lastpage}
\usepackage{hyperref}
\usepackage{gensymb}
\usepackage[absolute]{textpos}

\title{Exploring Spectral Analysis: PSET~5}
\author{
        Benjamin Getraer \\
        GEO~422:~Data, Models, \& Uncertainty in the Natural Sciences\\
}
\date{\today}



\begin{document}
\maketitle

\begin{abstract}
This is the paper's abstract \ldots
\end{abstract}

\section{Introduction}
Introduction

\paragraph{Outline}
The remainder of this article is organized as follows.
Section~\ref{sec:Nyquist} introduces the implications of sampling and aliasing.
Our new and exciting results are described in Section~\ref{results}.
Finally, Section~\ref{conclusions} gives the conclusions.

\section{Nyquist Sampling}\label{sec:Nyquist}
Nyquist sampling follows the theorem that for a given signal $f(\rm{x})$ composed of frequencies $\underline{\omega}$, $f(\rm{x})$ can be recovered using a sampling rate of $\frac{1}{\Delta\rm{x}}\ge2\times \omega_{\text{max}}$.
\begin{figure}[h!]
\centering

\includegraphics[height=0.6\linewidth]{/Users/benjamingetraer/Documents/Fall2017/GEO422/Figures/PSET5/Nyquist.pdf}

\caption{Sampling (in red) of the signal $f(\rm{x})=\cos(2\pi\times\rm{x}) +\sin(2\pi\times2\rm{x}) +\sin(2\pi\times5.6\rm{x})$ (in blue) at various sampling rates, where ``Nyquist Sampling'' is defined as a sampling rate of $\frac{1}{\Delta\rm{x}}=2\times5.6$.}
\end{figure}





\section{Results}\label{results}
In this section we describe the results.

\section{Conclusions}\label{conclusions}
We worked hard, and achieved very little.

\end{document}
