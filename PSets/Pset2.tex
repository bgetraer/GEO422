\documentclass[12pt]{article}
% packages that I call bc they are in the template I have and I don't know if I want them or not.
\widowpenalty=10000 
\clubpenalty=10000 
\usepackage{amsmath}
\usepackage{amsfonts}
\usepackage{amssymb}
\usepackage{wasysym}
\usepackage{graphicx}
\usepackage{pslatex}
\usepackage{lscape}
\usepackage{rotating}
\usepackage[T1]{fontenc}
\usepackage[latin1]{inputenc}
\usepackage{longtable}
\usepackage[font=scriptsize,labelfont=bf]{caption}
\setlength{\LTcapwidth}{5.5 in}
\usepackage{url}
\usepackage{lastpage}
\usepackage{lineno}
\usepackage[english]{babel}
\usepackage[usenames, dvipsnames]{color}
\usepackage[round, sort, numbers, authoryear]{natbib}
\usepackage{hyperref}
\usepackage{gensymb}
\hypersetup{colorlinks,%
citecolor=black,%
filecolor=black,%
linkcolor=Mahogany,%
urlcolor=MidnightBlue,%
pdftex}

\usepackage{fancyhdr}
\pagestyle{fancy}
 \rhead {\emph{\textcolor{blue}{bgetraer}, \thepage ~of
     \pageref{LastPage}}}

 \lhead{\footnotesize \textsc{GEO~422$/$PSET~2}}
 
 \cfoot{}
 \renewcommand{\headrulewidth}{0.4pt} 





\begin{document}

\section{Convolution of Uniform Distributions}

\begin{figure}[h!]
\centering
\includegraphics[width=1\textwidth]
{/Users/benjamingetraer/Documents/Fall2017/GEO422/Figures/PSET2/Fig1.pdf}
\caption[]{Convolutions of uniform pdf (red) compared to the Gaussian pdf (black). All pdf's are adjusted for $\mu=0$ and $\sigma^2=1$. Note the clear convergence of the convolved pdf to the Gaussian pdf, demonstrating the Central Limit Theorem.} \label{fig:ConvUnif}
\end{figure}



\newpage
\section{Convolution of Non-Uniform Distributions}

\begin{figure}[h!]
\centering
\includegraphics[width=1\textwidth]
{/Users/benjamingetraer/Documents/Fall2017/GEO422/Figures/PSET2/Fig21.pdf}
\quad
\includegraphics[width=1\textwidth]
{/Users/benjamingetraer/Documents/Fall2017/GEO422/Figures/PSET2/Fig22.pdf}
\caption[]{Convolutions of non-uniform pdf's (red) compared to the Gaussian pdf (black). Top: A semi-circular pdf convolved with a step-shaped pdf; Bottom: An exponential-shaped pdf convolved with a house-shaped pdf. Note that despite having different and somewhat arbitrary shape and location, there is still convergence to the Gaussian pdf after 30 convolutions.} \label{fig:ConvNonUnif}
\end{figure}
\newpage
\section{$\chi^2$ Test Against ``TRUE'' Distribution}
Demonstration of $\chi^2$ test of whether random samples of a normally distributed population came from that ``TRUE'' population. Sample frequency in binned areas was compared to ``TRUE'' frequency of the area under the pdf curve within those bin edges (Figure~\ref{fig:ChiTrue1}) and a $\chi^2$ value was calculated using {\bf equation~(6)}.

\begin{figure}[h!]
\centering
\includegraphics[width=0.7\textwidth]
{/Users/benjamingetraer/Documents/Fall2017/GEO422/Figures/PSET2/Fig31.pdf}
\caption[]{Example of samples drawn (blue histogram) compared to the pdf from which they were taken (red line).} \label{fig:ChiTRUE1}
\end{figure}

The test is done by comparing the $\chi^2$ value from our samples to the $\chi^2_{k-3}$ pdf. The probability of obtaining a higher $\chi^2$ value than the one observed is calculated from the area under the curve of the $\chi^2_{k-3}$ pdf from the $\chi^2$ to $\infty$ (Figure~\ref{fig:ChiTRUE2}).

\begin{figure}[h!]
\centering
\includegraphics[width=0.7\textwidth]
{/Users/benjamingetraer/Documents/Fall2017/GEO422/Figures/PSET2/Fig32.pdf}
\caption[]{Example an observed $\chi^2$ value compared to the $\chi^2_{k-3}$ pdf, with the shaded area representing the probability used in the test.} \label{fig:ChiTRUE2}
\end{figure}

SOME RESULTS FROM (3):
\\Gaussian compared to TRUE distribution: $1000$ samples
\\fraction that pass at $95\%$ confidence
\\	$0.7540$
\\	$0.7550$
\\	$0.7640$
\\	$0.7490$
\\
\\" " " at $80\%$ confidence
\\	$0.5340$
\\	$0.5080$
\\	$0.5420$
\\	$0.5440$


\newpage
\section{$\chi^2$ Test Against ``FALSE'' Distribution}
Demonstration of $\chi^2$ test of whether random samples pulled from a Rayleigh pdf came from a ``FALSE'' population (i.e. a Gaussian pdf of the same $\mu$ and $\sigma^2$). Sample frequency in binned areas was compared to ``FALSE'' frequency of the area under the pdf curve within those bin edges (Figure~\ref{fig:ChiFALSE1}) and a $\chi^2$ value was calculated using {\bf equation~(6)}.

\begin{figure}[h!]
\centering
\includegraphics[width=0.7\textwidth]
{/Users/benjamingetraer/Documents/Fall2017/GEO422/Figures/PSET2/Fig44.pdf}
\caption[]{Example of samples drawn (blue histogram) compared to the ``FALSE'' pdf from which they were NOT taken (red line).} \label{fig:ChiFALSE1}
\end{figure}

SOME RESULTS FROM (4)
\\Rayleigh distribution compared to Gaussian: $95\%$ confidence
\\$50$ samples
\\	$0.5060$
\\	$0.4760$
\\	$0.4940$
\\$75$ samples
\\	$0.4690$
\\	$0.4480$
\\	$0.4400$
\\$100$ samples
\\	$0.3540$
\\	$0.3750$
\\	$0.3550$
\\$250$ samples
\\	$0.1410$
\\	$0.1310$
\\	$0.1210$
\\$500$ samples
\\	$0.0210$
\\	$0.0180$
\\	$0.0190$
\\$750$ samples
\\	$0.0030$
\\	$0$
\\	$0.0020$
\\$1000$ samples
\\	none pass


\end{document}


This is never printed
